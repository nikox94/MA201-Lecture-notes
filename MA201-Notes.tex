\documentclass[11pt]{article}
\usepackage[a4paper,includeheadfoot,margin=2.54cm]{geometry}
\usepackage[T2A]{fontenc}
\usepackage[utf8]{inputenc}
\usepackage[bulgarian]{babel}
\usepackage[unicode=true]{hyperref}
\usepackage{breakurl}
\usepackage{indentfirst}
\usepackage{amsmath}
\usepackage{amsthm}
\usepackage{amssymb}
\usepackage{mathtools}
\usepackage{esvect}

\numberwithin{equation}{section}
\numberwithin{figure}{section}
\numberwithin{table}{section}
  \theoremstyle{plain}
  \newtheorem{thm}{\protect\theoremname}[section]
  \theoremstyle{definition}
  \newtheorem{defn}[thm]{\protect\definitionname}
  \theoremstyle{remark}
  \newtheorem*{notation*}{\protect\notationname}
  \theoremstyle{definition}
  \newtheorem*{example*}{\protect\examplename}
  \theoremstyle{remark}
  \newtheorem*{note*}{\protect\notename}
  \theoremstyle{plain}
  \newtheorem{lem}[thm]{\protect\lemmaname}
  \theoremstyle{definition}
  \newtheorem*{defn*}{\protect\definitionname}
  \theoremstyle{definition}
  \newtheorem{example}[thm]{\protect\examplename}
  \theoremstyle{plain}
  \newtheorem{cor}[thm]{\protect\corollaryname}
  \theoremstyle{plain}
  \newtheorem{prop}[thm]{\protect\propositionname}
  \theoremstyle{plain}
  \newtheorem*{prop*}{\protect\propositionname}
  \theoremstyle{definition}
  \newtheorem{xca}[thm]{\protect\exercisename}
  \newcommand\thmsname{\protect\theoremname}
%  \newcommand\nm@thmtype{theorem}
  \theoremstyle{plain}
  \newtheorem*{namedtheorem}{\thmsname}
  \newenvironment{namedthm}[1][Undefined Theorem Name]{
   \ifx{#1}{Undefined Theorem Name}\renewcommand\nm@thmtype{theorem*}
   \else\renewcommand\thmsname{#1}\renewcommand\nm@thmtype{namedtheorem}
   \fi
   \begin{\nm@thmtype}}
   {\end{\nm@thmtype}}

  \providecommand{\corollaryname}{Следствие}
  \providecommand{\definitionname}{Дефиниция}
  \providecommand{\examplename}{Пример}
  \providecommand{\exercisename}{Упражнение}
  \providecommand{\lemmaname}{Лема}
  \providecommand{\notationname}{Нотация}
  \providecommand{\notename}{Забележка}
  \providecommand{\propositionname}{Твърдение}
  \providecommand{\remarkname}{Забележка}
  \providecommand{\theoremname}{Теорема}

\DeclarePairedDelimiter\norm{\lVert}{\rVert}
\renewcommand*{\Vec}[1]{\mathbf{#1}}
\newcommand*{\Z}{\Vec{0}}
\newcommand*{\B}{\mathcal{B}}
\newcommand*{\R}{\mathbb{R}}

\title{Лекционни записки по Математически Анализ}
\author{проф. Надежда Рибарска \\ Набрани от Никола Юруков}
\date{\today}


\begin{document}

\maketitle

\clearpage

\tableofcontents

\clearpage

\section{Лекция 1 - преговор с разширение}

\subsection{Евклидовото пространство $\mathbb{R}^n$}

Като множество $\mathbb{R}^n$ е множеството $\{x = (x_1, x_2, ..., x_n): x_i \in \mathbb{R}, \  i=1,2,..,n\}$ от нередените $n$-торки реални числа. Ако го снабдим със стандартните линейни операции събиране на вектори и умножение на вектор с реално число, получаваме реално линейно пространство (спомнете си аксиомите от курса по линейна алгебра). Да напомним формалните дефиниции: сума на векторите $x = (x_1, x_2, ..., x_n)$ и $y = (y_1, y_2, ..., y_n)$ е векторът $x+y = (x_1+y_1, x_2+y_2, ..., x_n + y_n)$ (събирането е покоординатно). Произведение на скалара $\lambda \in \mathbb{R}$ с вектора $x$ е векторът $\lambda x = (\lambda x_1, \lambda x_2, ..., \lambda x_n)$ (умножението със скалар също е покоординатно). Ще означаваме с $\Z$ нулевия вектор $(0,\dots ,0)$.

За да можем да правим анализ (да говорим за граница, непрекъснатост, производна и т.н.), освен линейната структура ни е необходима и някаква "мярка на близост" в нашето пространство. Както помните от курса по ДИС2, стандартната мярка на близост  между два вектора е евклидовото разстояние между тях:
 $$\rho(x,y):= \sqrt{\sum_{i=1}^n (x_i-y_i)^2} \ , \mbox{ където } x = (x_1, x_2, ..., x_n), \ y = (y_1, y_2, ..., y_n) \ .$$
 Забележете, че в $\mathbb{R}^2$ това е просто питагоровата теорема. Това разстояние е добре съгласувано с линейната структура в смисъл, че $\rho(x,y)=\lVert x - y\rVert$, където в дясната част стои евклидовата норма (или дължината) на вектора $x-y$:
 $$\lVert x\rVert := \sqrt{\sum_{i=1}^n x_i^2}\ , \ x = (x_1, x_2, ..., x_n) \ .$$

 Да напомним, че една функция $\lVert \cdot \rVert :\mathbb{R}^n \longrightarrow [0,+\infty)$ се нарича норма, ако за нея са в сила свойствата
\begin{enumerate}
	\item $\lVert x \rVert = 0 \iff x=\Z$
	\item $\norm{\lambda x} = |\lambda|\cdot \norm{x}$
	\item $\norm{x+y} \leq \norm{x} + \norm{y}$ (неравенство на триъгълника)
\end{enumerate}

В курса по ДИС2 е проверено, че евклидовата норма е норма. За упражнение проверете, че
\begin{itemize}
 \item $\norm{(x_1, x_2)}_1 = |x_1| + |x_2|$
 \item $\norm{(x_1, x_2)}_\infty = $ max\{$|x_1|, |x_2|$\}
 \item $\norm{(x_1, x_2)}_p = \sqrt[p]{|x_1|^p + |x_2|^p}$, $1< p<\infty$
\end{itemize}
са норми в $\mathbb{R}^2$.
По-общо, проверете, че  $$\norm{x}_p = \sqrt[p]{\sum_{i=1}^n |x_i|^p} \ , \ 1\leq p<\infty \mbox{ е норма в } \mathbb{R}^n \ .$$
Разбира се, за целта трябва да използвате неравенството на Минковски от курса по ДИС2.

Евклидовата норма има по-хубави геометрични свойства от горните примери, защото е съгласувана със скаларното произведение
$$\langle x, y\rangle = \sum_{i=1}^n x_i y_i  \ , \mbox{ където } x = (x_1, x_2, ..., x_n) \mbox{ и } \ y = (y_1, y_2, ..., y_n),$$
по стандартния начин $\lVert x \rVert =\sqrt{\langle x, x\rangle}$. Да напомним основното неравенство на Коши-Буняковски-Шварц:
$$|\langle x, y\rangle | \leq \norm{x}\norm{y} \ .$$

Да напомним също означенията
$$B_r(x) := \left\{ y\in \mathbb{R}^n : \norm{y-x}<r\right\}$$
за отворено кълбо с център $x$ и радиус $r$ и
$$\overline{B}_r(x) := \left\{ y\in \mathbb{R}^n : \norm{y-x}\leq r\right\}$$
за затворено кълбо с център $x$ и радиус $r$. Като упражнение можете да скицирате кълбата с радиус 1 и център началото на координатната система за нормите $\norm{\cdot}_1$ и $\norm{\cdot}_\infty$ от предишното упражнение.

%TODO add graphics for intervals and p-norms.

\subsection{Топология в $\mathbb{R}^n$}

\begin{defn}
Подмножеството $U$ на $\R^n$ се нарича отворено, ако за всяка точка $x$ от $U$ съществува $\epsilon>0$ такова, че $B_\epsilon (x) \subset U$.
\end{defn}

Основните свойства на отворените множества, проверени в курса по ДИС2, са

1. $\emptyset$ и $\R^n$ са отворени

2. Сечение на краен брой отворени множества е отворено, т.е. ако $U_1, U_2, ..., U_k$ са отворени, то $\bigcap_{i=1}^k U_i$ е отворено.

3. Обединение на произволна фамилия от отворени множества е отворено, т.е. ако $U_\alpha$ са отворени за всяко $\alpha \in I$, то $\bigcup _{\alpha \in I} U_\alpha$ е отворено.

\begin{example}
Отворените кълба са отворени множества.\\
Да разгледаме $B_r(x_0)$, $r>0$. Взимаме си произволно $x$ от кълбото, т.е. растоянието между $x$ и $x_0$ е по-малко от $r$. Нека $\epsilon := r - \norm{x_0-x}>0$. Тогава $B_\epsilon(x)\subset\B_r(x_0)$. Наистина, нека $y\in B_\epsilon(x)$, т.е. $\norm{y-x} <\epsilon$. Получаваме
\begin{eqnarray*}
\norm{x_0-y} &\leq& \norm{x-y} + \norm{x-x_0} < \epsilon + \norm{x-x_0}\\
\norm{x_0-y} &<& r - \norm{x_0-x} + \norm{x-x_0}\\
\norm{x_0-y} &<& r
\end{eqnarray*}
\end{example}

\begin{example}
Нека функцията  $g:\R^n \rightarrow \R$ е \textbf{непрекъсната}. Тогава множеството \\ $U = \{x\in\R^n:g(x)>0\}$ е отворено.
\begin{proof}
Взимаме произволна точка $x_0\in U$, следователно $\epsilon = g(x_0)>0$. От непрекъснатостта на функцията получаваме, че съществува положително число $\delta$ такова, че $|g(x)-g(x_0)|<\epsilon$ за всяко $x\in B_\delta(x_0)$. Следователно $g(x)>g(x_0)-\epsilon = 0$ и оттук $x \in U$ за всяко $x\in B_\delta(x_0)$.
\end{proof}
\end{example}

\begin{defn} Едно подмножество
$F$ на $\R^n $ се нарича затворено, ако $\R^n \setminus F$ е отворено множество.
\end{defn}

Основните свойства на затворените множества, проверени в курса по ДИС2, са

1. $\emptyset, \R^n$ са затворени.

2. Обединие на краен брой затворени множества е затворено, т.е. ако $F_1, F_2, ..., F_k$ са затворени, то $\bigcup_{i=1}^k F_i$ е затворено.

3. Сечение на произволна фамилия от затворени множества е затворено, т.е. ако $F_\alpha$ са затворени за всички $\alpha \in I$, то $\bigcap _{\alpha \in A} F_\alpha$ е затворено.


\noindent Пример: Затворените кълба са затворени множества.

Да напомним още едно свойство на затворените множества, доказано в ДИС2: Едно множество $F$ е затворено точно тогава, когато $F$ съдържа границите на всички редици, съставени от негови елементи. Иначе казано, 
$$F = \left\{ x\in \R^n : \ \exists \{x_m\}_{m=1}^\infty \subset F, \ x_m \rightarrow x \right\} \ .$$

% UNREVIEWED TEXT BEGINS HERE

\begin{defn}
Контур на множество.

\noindent Нека $A \subset \R^n$. Тогава контур на $A$ наричаме множеството
$$\partial A = \{ x\in \R^n : x\in U, \; \forall U \; \text{отворено} \land U \cap A \neq \emptyset \land M \setminus A \neq \emptyset \}$$
%Горе май има грешка... M\A != \emptyset ... mmm... 
Също можем да напишем
$$\partial A = \{x \in \R^n \text{, за които същ. } \{x_m\}_{m=1}^\infty \subset A, x_m \rightarrow x \text{, същ. } \{y_m\}_{m=1}^\infty \subset \R^n\setminus A, y_m \rightarrow x\}$$
\end{defn}

$A \rightarrow \partial A$ е затворено множество

\begin{defn}
Затворена обвивка на множество.
$$\overline{A} = A \cup \partial A = \cap F (F \text{затв} \supset A)$$
Най-малкото затворено съдържащо $A$.
$$\overline{A} = \{x\in\R^n: \exists \{x_m\}_{m=1}^\infty \subset A,\; x_m \rightarrow x \}$$
Контурът е затворена обвивка на множеството пресечена със затворената обвивка на допълнението.
\end{defn}

\begin{defn}
Вътрешност на $A\subset\R^n$.
$$\mathring{A} = \bigcup_{\substack{U \text{отв.}\\ U \subset A}} U = \R^n \setminus \overline{(\R^n\setminus A)} = int(A)$$
Обединението на отворените множества, които се съдържат в $A$.
\end{defn}

\begin{defn}
Компактност.

\noindent $A\subset \R^n$ се нарича компакт $\iff A$ е ограничено и затворено.

От всеки ред от негови елементи може да се избере сходяща подредица, чиято граница е също в множеството.
$$\forall\{x_m\}_{m=1}^\infty \subset A \; \exists x_{m_k} \xrightarrow[k\rightarrow \infty]{} x_0 \in A$$
Каквото и отворено покритие на $A$ да вземем, можем да изберем негово крайно подпокритие.
$$\forall \{U_\alpha\}_{\alpha \in I} : U_\alpha \text{ са отворени и } \cup_{\alpha\in I} U_\alpha \supset A, \text{тогава} \; \exists \alpha_1, \alpha_2, ..., \alpha_k \in I : \cup_{i=1}^k U_{\alpha_i} \supset A$$
\end{defn}

\subsection{Основни теореми}

%Тази теорема е доста оскъдна в записките. Хубаво е да се понапълни.
\begin{thm}[Теорема на Вайерщрас]
Непрекъснат образ на компакт е компакт.
$$\text{непр. } f: (K \subset \R^n) \rightarrow \R^m \Rightarrow f(K) = \{f(x):x\in K\} \text{ е компакт в }\R^m$$
\end{thm}
\begin{proof}
Взимаме $\{y_l\}_{l=1}^\infty \subset f(K)$ и търсим нейна сходяща подредица. $y(l) = f(x_l), \; x_l \in K$. $\{x_l\}_{l=1}^\infty \subset K$ компакт. $x_{l_k} \xrightarrow[k \rightarrow \infty]{} x_0 \in K$.

$f$ е непрекъсната $\Rightarrow f(x_{l_k}) = y_{l_k} \xrightarrow[k \rightarrow \infty]{} f(x_0)\in f(K)$
\end{proof}

\begin{thm}[Теорема на Кантор]
Нека $f$ е дефинирана в $D \subset \R^n$. Нека $K$ компакт $\subset D$.
$f$ е непрекъсната в $K$, т.е. непрекъсната във всяка точка от $K$.
Тогава твърдим, че $f$ е равномерно непрекъсната в $K$. Тоест
$$\forall \epsilon >0 \exists \delta >0 : \forall x \in K \quad \forall x' \in D \quad \norm{x'-x}<\delta$$
да е в сила
$$|f(x)-f(x')|<\epsilon$$
\end{thm}

\begin{proof}
Допускаме противното.

$\exists \epsilon_0 >0 \; \forall \delta >0 \; \exists x_\delta \in K$ зависещо от $\delta$, $\exists x'_\delta \in D, \norm{x_\delta - x'_\delta}<\delta : |f(x)-f(x')|\geq \epsilon$
Даваме стойности на $\delta = 1, \frac{1}{2}, \frac{1}{3}, ...$, за да клони към 0. Така се образуват две редици: $\{x_m\}_{m=1}^\infty \subset K$ и $\{x'_m\}_{m=1}^\infty \subset D$.
Знаем, че разстоянието $< \delta = \frac{1}{m}$, тоест $\norm{x_m - x'_m}<\frac{1}{m}$ и
\begin{equation} \label{eq:K1}
|f(x_m) - f(x'_m)|\geq \epsilon_0 >0
\end{equation}
$K$ е компакт $\Rightarrow \exists $ сходяща подредица $x_{m_k} \xrightarrow[k \rightarrow \infty]{} x_0 \in K$ от непрекъснатостта на $f$. Следователно $f(x_{m_k} \xrightarrow[k \rightarrow \infty]{} f(x_0)$. И така съществува подредица с примове, такава че $\norm{x'_{m_k}-x_0}\leq \norm{x'_{m_k}-x_{m_k}} + \norm{x_{m_k} - x_0} < \frac{1}{m} + \norm{x_{m_k}-x_0} \xrightarrow[k \rightarrow \infty]{} 0$. Следователно $x'_{m_k} \xrightarrow[k \rightarrow \infty]{} x_0 \in K \Rightarrow f(x'_{m_k}) \xrightarrow[k \rightarrow \infty]{} f(x_0)$.
\begin{equation} \label{eq:K2}
f(x_{m_k}) - f(x'_{m_k}) \xrightarrow[k \rightarrow \infty]{} 0
\end{equation}
Така от \eqref{eq:K1} и \eqref{eq:K2} се получава противоречие. Следователно теоремата е доказана.
\end{proof}

\begin{defn}
Релативно отворено множество.
Нека $A \subset \R^m$. $U\subset A$ наричаме релативно отворено в $A$, ако съществува множество отворено в цялото пространство ($V\subset \R^m$), такова че $U = A\cap V$.
\end{defn}

\begin{prop}
Нека $f: (D\subset \R^n) \rightarrow \R^m$. Ако $f$ е непрекъсната в $D \iff \forall U \text{ отворено } \subset \R^m: f^{-1} (U) = \{x\in D: f(x) \in U\}$ е релативно отворено в $D$.
\end{prop}

\begin{proof}
Първо ще докажем обратната посока $\Leftarrow$.

Избираме точка от D ($x\in D$) и $\epsilon_0 >0$ Взимаме първообраз $f^-1 (\B_{\epsilon_0}(f(x)) = D \cap V \Rightarrow x \in V$ отворено и $\exists \delta >0 : \B_\delta (x) \subset V$. Сега $x' \in D \cap \B_\delta(x) \Rightarrow x' \in D\cap V \Rightarrow f(x') \in \B_{\epsilon_0}(f(x))$.

Сега обратната посока $\Rightarrow$.
Взимаме $U \subset \R^n$ и $x\in f^{-1} (U) \Rightarrow f(x) \in U$ отворено. Следователно $\exists \epsilon >0 ; \B_\epsilon(f(x)) \subset U$, $f$ е непрекъсната $\Rightarrow \exists \delta_x >0, f(\B_{\delta_x}(x)\cap D) \subset \B_\epsilon (f(x)) \subset U$. И така нека имаме отвореното множество $V = \bigcup_{x \in f^{-1}(U)} \B_{\delta_x}(x)$. Имаме, че $V\cap D = f^{-1} (U)$, защото:
\begin{enumerate}
	\item $\supset$, защото е център.
	\item $\subset$, защото $y \in V \cap D$, $y \in D \; y\in \B_{\delta_x}(x)$, $\Rightarrow f(y) \in f(\B_{\delta_x}(x)\cap D) \subset \B_\epsilon (f(x)) \subset U$ и следователно $y \in f^{-1} (U)$. 
\end{enumerate}
\end{proof}
За упражнение да се докаже теоремата на Вайерщрас, използвайки горното твърдение.

\end{document}